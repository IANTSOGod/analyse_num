\documentclass[a4paper,12pt]{report}

\usepackage[utf8]{inputenc}   % Encodage UTF-8
\usepackage[T1]{fontenc}
\usepackage[french]{babel}
\usepackage{graphicx}         % Pour insérer des images
\usepackage{amsmath, amssymb} % Pour les maths
% \usepackage{hyperref}         % Pour les liens cliquables
\usepackage{geometry}         % Marges du document
\usepackage{palatino}
\usepackage[colorlinks=true, linkcolor=black, urlcolor=blue, citecolor=blue]{hyperref}



\geometry{margin=2.5cm}

\begin{document}
\begin{titlepage}
    \centering
    % --- Logo en haut ---
    \includegraphics[width=0.12\textwidth]{univ.png}\hspace{1cm}
    \includegraphics[width=0.12\textwidth]{fac.jpg}\hspace{1cm}
    \includegraphics[width=0.12\textwidth]{IT.jpg}\\[1.5cm]

    % --- Infos Université ---
    {\large Université d'Antananarivo \\ 
    Faculté des Sciences \\ [0.1cm]
    Mention Informatique et Technologie}\\[1.5cm]

    % --- Grand titre ---
    {\Huge \bfseries Rapport de Projet de Réseau de Neurones}\\[0.5cm]
    {\large Prédiction de la langue d'une phrase avec un réseau de neurones}\\[2cm]

    % --- Auteurs ---
    \textbf{Groupe de 4 personnes :}\\
    Iantso, Fandresena, Brice, Tsito\\[1cm]

    % --- Date ---
    {\large \today}
\end{titlepage}

\tableofcontents
\newpage

\chapter*{Introduction}
\addcontentsline{toc}{chapter}{Introduction}
Dans ce projet, nous nous intéressons à la reconnaissance automatique de la langue d'une phrase.
L'objectif est de développer un modèle de réseau de neurones capable de prédire la langue (par exemple : français, anglais, espagnol, etc.) à partir d'une entrée textuelle.

\section{Contexte}
La détection automatique de la langue est une tâche importante dans le traitement automatique du langage naturel (TALN).
Elle est utilisée dans :
\begin{itemize}
	\item Les traducteurs automatiques (Google Translate, DeepL, etc.)
	\item Les systèmes de reconnaissance vocale
	\item Les moteurs de recherche multilingues
\end{itemize}

\section{Objectifs}
\begin{itemize}
	\item Concevoir et entraîner un réseau de neurones simple pour classer les phrases selon leur langue.
	\item Évaluer la performance du modèle sur un jeu de test.
	\item Comparer avec des approches classiques de détection de langue.
\end{itemize}

\chapter*{Méthodologie}
\addcontentsline{toc}{chapter}{Méthodologie}

\section{Données utilisées}
Pour entraîner le modèle, nous avons utilisé un dataset contenant des phrases dans plusieurs langues (français, anglais, espagnol, etc.).
Chaque phrase est associée à une étiquette correspondant à la langue.\\ Malheureusement nous n'avons pas trouvés de dataset avec le langue malgache

\section{Architecture du réseau de neurones}
Le modèle utilisé est un réseau de neurones simple codé avec le langage de programmation python avec :
\begin{itemize}
	\item Couche d'entrée : représentation vectorielle des phrases.
	\item Couches cachées : couches entièrement connectées avec fonction d'activation ReLU.
	\item Couche de sortie : classification multiclasse avec fonction softmax.
\end{itemize}

L'équation générale de la couche dense est :
\[
	y = f(Wx + b)
\]

où $W$ est la matrice de poids, $b$ le biais, et $f$ la fonction d'activation.\\

L'equation de la fonction ReLU est :

\[
	ReLU(x)=max(0,x)
\]\\

Et l'equation de la fonction softmax est :\\

\[
\text{Softmax}(z_i) = \frac{e^{z_i - \max_j z_j}}{\sum_{k=1}^{n} e^{z_k - \max_j z_j}} \quad \text{pour } i = 1, 2, \dots, n
\]\\

$zi$: la valeur du i‑ème élément du vecteur d’entrée

$n$ : le nombre total de classes

et la somme des Softmax sur toutes les classes vaut 1 :
\[
\sum_{i=1}^{n} \text{Softmax}(z_i) = 1
\]\\

\section{Prétraitement des données}
Avant d'entrer dans le réseau, les phrases ont été vectorisé puis normalisé

\section{Hyperparamètres du modèle}
\begin{itemize}
    \item Nombre de couches cachées : 3
    \item Neurones par couche : 512, 256, 128
    \item Fonction d'activation : ReLU
    \item Fonction de perte : Cross-Entropy
    \item Batch size : 32
    \item Nombre d'époques : 500
\end{itemize}





% \chapter*{Résultats et discussion}
% Après entraînement sur plusieurs époques, le modèle atteint une précision de 92\% sur l'ensemble de test.
% Les erreurs se produisent principalement entre les langues proches (exemple : espagnol vs portugais).

% \section{Matrice de confusion}


\end{document}
